\documentclass{article}
\usepackage[utf8]{inputenc}

\title{Documentazione Progetto HEADER}
\author{Dridi Dario \and Marianelli Manuele \and Testi Guido}
\date{\today}

\begin{document}

\maketitle

\section{Presentazione}
Il nostro algoritmo legge dal file \texttt{msg.txt} il messaggio e lo memorizza in una variabile di tipo stringa nel \texttt{main}; ogni carattere della stringa viene inviato come parametro alla funzione di conversione in binario che restituisce la conversione in binario della lettera passata; ogni conversione di ogni lettera viene aggiunta a un’altra variabile che successivamente sarà scritta nel file \texttt{frame.txt}. Inoltre abbiamo dichiarato una \texttt{struct} con tutti i campi dell’header IP e in un’altra procedura li abbiamo impostati.

\section{File del progetto}
\subsection{main.cpp}
All’interno del file \texttt{main.cpp} è presente esclusivamente l’inclusione delle librerie e il codice all’interno del \texttt{main}; qui sono presenti le dichiarazioni dei parametri attuali e le chiamate di funzioni e procedure. Per la conversione del messaggio in binario abbiamo memorizzato l’intera riga in \texttt{msg.txt} in una stringa e poi abbiamo chiamato la funzione di conversione per ogni carattere della stringa.

\subsection{funzioni.cpp}
All’interno del file \texttt{funzioni.cpp} è presente il corpo di tutte le funzioni, procedure utilizzate e l’attuale \texttt{struct} utilizzata.

\subsection{funzioni.h}
All’interno del file \texttt{funzioni.h} sono presenti i prototipi delle funzioni e procedure (scritte in \texttt{funzioni.cpp}) e la dichiarazione della \texttt{struct IpHeader} con tutti i campi.

\subsection{msg.txt}
All’interno del file \texttt{msg.txt} è presente il messaggio di partenza del mittente.

\subsection{frame.txt}
All’interno del file \texttt{frame.txt} è presente la conversione in binario del messaggio scritto in \texttt{msg.txt}.

\end{document}
